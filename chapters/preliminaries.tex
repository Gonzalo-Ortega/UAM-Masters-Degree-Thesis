\chapter{Preliminaries}
The contents of this chapter are based on \cite{nanda}, \cite{polterovich} and \cite{wang}.

\begin{definition}[Graded ring]
    Let $ R $ be a ring. It is said that $ R $ is a {\bf graded ring} if it can be decomposed into a direct sum of additive groups
    $$
        R = \bigoplus_{n=1}^{\infty} R_n = R_1 \oplus R_2 \oplus \dots
    $$
    such that for all $ n, m \geq 0 $, 
    $$
        R_n R_m = R_{n+m}.
    $$
\end{definition}

\begin{definition}[Graded ideal]
    Let $ R $ be a graded ring. A {\bf graded ideal} is a two sided ideal $ I \subseteq R $ that can be decomposed into a direct sum
    $$
        I = \bigoplus_{n=1}^{\infty} I_n
    $$
    where each $n \geq 0 $, $ I_n = I \cap R_n $.
\end{definition}

\begin{definition}[Left moudule, Definition IV.1.1.1 \cite{hungerford}]
    Let $ R $ be a ring. A {\bf left $R$-module } is an abelian group $ (M, +) $ with an operation $ \cdot \colon R \times M \to M $ such that for all $ r, s \in R $ and for all $ x, y \in M $,
    \begin{enumerate}
    \renewcommand{\labelenumi}{(\roman{enumi})}
        \item $ (rs) \cdot x = r (s \cdot x) $,
        \item $ (r + s) = r \cdot x + s \cdot x $,
        \item $ r \cdot (x + y) = r \cdot x + r \cdot y $.
    \end{enumerate}
    If $ R $ has a multiplicative identity $ 1 $, then $ M $ is said to be a {\bf unitary $R$-module} and 
    \begin{enumerate}
    \renewcommand{\labelenumi}{(\roman{enumi})}
        \setcounter{enumi}{3}
        \item $1 \cdot x = x $.
    \end{enumerate}
    If $ R $ is a division ring, that is, a ring with identity where every non cero element is a unit, then 
    a unitary $R$-module is called a {\bf left vector space}. Note that in this case, $R$ is in fact a field.
\end{definition}

\begin{definition}[Graded moudule, Definition 4.7 \cite{wang}]
    Let $M$ be a left module over a graded ring $ R $. It is said that $ M $ is a {\bf left graded module} if it can be decomposed into a direct sum
    $$
        M = \bigoplus_{n=1}^{\infty} M_n
    $$
    if for each $n, m \geq 0 $, $ R_n M_m \subseteq M_{n+m} $.
\end{definition}

\begin{definition}[Persistance module]
    Let $ V = \{V_t\}_{t \in \mathbb R} $ is a collection of finite dimensional vector spaces over a field $ \mathbb F $. A {\bf persistence module} is a pair $ (V, \pi) $ such that $ \pi = \{ \pi_{s \leq t} \} $ is a collection of linear maps $ \pi_{s \leq t}\colon V_s \rightarrow V_t $ that verifies
    \begin{enumerate}
    \renewcommand{\labelenumi}{(\roman{enumi})}
    \item a
    \end{enumerate}
\end{definition}

\begin{definition}[Module morphism, shift]
    
\end{definition}

\begin{definition}[Interval module]
    
\end{definition}

\begin{definition}[Direct sum of persistance modules]
    
\end{definition}

\begin{definition}[Barcode]

\end{definition}

\begin{definition}[$\delta$-interleaving moudules]
    
\end{definition}

\begin{definition}[Interleaving distance]
    
\end{definition}

\begin{definition}[Multiset matching]
     
\end{definition}

\begin{definition}[$\delta$-matching barcodes] \label{delta-matching}
    
\end{definition}

\begin{definition}[Bottleneck distance]
    
\end{definition}
