\chapter{Gromov-Hausdorff's Stability Theorem}
\section{Gromov-Hausdorff stability}

\begin{lemma}[Exercise 3.5.4, \cite{burago}] \label{lemma:haus-aux-1}
    Any finite metric space of cardinality $ n $ can be isometrically embedded into $(\R^n, \ell^n)$.
\end{lemma}
\begin{proof}
    Let $ X $ be a compact metric space and let $ C(X) $ be the space of all continuous functions from $ X $ to $ \R $. Let $ f, g \in C(X) $. Recall the uniform distance given by
    \begin{equation}
        d_\infty(f, g) \colon \sup | f(x) - g(x) |.
    \end{equation}
    First, we will check that the pair $ (C(X), d_\infty) $ is a metric space. Naturally nonnegativity holds and 
    \begin{equation}
        d_\infty(f, f) = \sup | f(x) - f(x) | = 0.
    \end{equation}
    Commutativity also holds as 
    \begin{equation}
        d_\infty(f, g) = \sup | f(x) - g(x) | = \sup | g(x) - f(x) | = d_\infty(g, f).
    \end{equation}
    Finally, if $ h \in C(X) $, triangle inequality holds because
    \begin{align}
        d_\infty(f, h) &= \sup | f(x) - h(x) | = \sup | f(x) + g(x) - g(x) - h(x) | \\
        &\leq \sup | f(x)- g(x) | + \sup | g(x) - h(x) | = d_\infty(f, h) + d_\infty(h, g).
    \end{align}
    Now we are going to verify that the map $ E \colon X \to C(X) $ defined by $ E(x) = d(x, \cdot) $ is an isometric embedding onto its image. Note that
    \begin{equation}
        d_\infty(d(x, \cdot), d(y, \cdot)) = \sup_z |d(x, z) - d(y, z)| \leq  \sup_z |d(x, y)| = d(x, y).
    \end{equation}
    On the other hand if we take $ z = y $ we then have
    \begin{equation}
        |d(x, y) - d(y, y)| = d(x, y),
    \end{equation}
    and therefore
    \begin{equation}
        \sup_z |d(x, z) - d(y, z)| \geq = d(x, y).
    \end{equation}
    The proof of the lemma is just an analogous case taking $ C_n(X) $ as the set of continuous functions $ f \colon X \to \R^n $, and, for every $ x = (x_1, \dots, x_n), y = (y_1, \dots, y_n) \in \R^n $, $ \ell^\infty(x, y) = \max_i |x_i - y_i|$.
\end{proof}

\begin{lemma}[Lemma VII, \cite{ghrist}] \label{lemma:haus-aux-2}
    Let $ X \subset \R^n $ and $ \alpha > 0 $. Then the $\alpha$-\v Cech and the $\alpha$-Rips complexes coincide when using the $ \ell^\infty $-norm. That is
    $$
        \check C_\alpha(X, \ell^\infty) = R_\alpha(X, \ell^\infty).
    $$
\end{lemma}

\begin{definition}[Paracompact space]
    Let $ X $ be a topological space. It is said to be {\bf paracompact} if for all covering $ U $ of $ X $, there exists $ \mathcal V \subseteq \mathcal U $ such that $ \mathcal V $ is a finite covering.
\end{definition}

\begin{definition}[Good cover]
    Let $ S $ be a topological space and $ I $ a set of indexes. A {\bf good cover} of $ S $ is a family $ \mathcal U = {U_i} $ of open subsets covering $ S $ such that for every finite subset $ J \subset I $, the common intersection
    \begin{equation}
        \bigcap_{j\in J} U_j
    \end{equation}
    is either empty or contractible.
\end{definition}

\begin{lemma}[\cite{oudot}] \label{lemma:haus-aux-3}
    Let $ S \subset S' $ be two paracompact spaces. Let $ \mathcal U = \{U_x\}_{x \in A}$, $ \mathcal U' = \{U'_x\}_{x \in A'}$ be two good covers of $ S $ and $ S'$ respectively, based on finite parameter sets $ A \subset A ' $ such that $ U_x \subset U_x' $ for all $ x \in A $. Then the homotopy equivalences $ \n \mathcal U \to S $ and $\n U' \to S' $ commute with the canonical inclusions $ S \to S' $ and $ \n \mathcal U \to \n \mathcal U' $ at homology level.
\end{lemma}

\begin{theorem}[Theorem 3.1, \cite{chazal}]
    Let $ (X, d_X) $, $ (Y, d_Y) $ be finite metric spaces. Then, for any $ k \in \N$, 
    \begin{equation}
        \db(  (\rf(X, d_X)), D_k(\rf(Y, d_Y))) \leq \dgh((X, d_X), (Y, d_Y)).
    \end{equation}
\end{theorem}
\begin{proof}
    Let $ \e = \dgh((X, d_X), (Y, d_Y)) $. As $ X $ and $ Y $ are finite, they are compact, and therefore the infimum when computing Gromov-Hausdorff distance is in fact a minimum. That is, there exists a metric space $ (Z, d_Z) $ and two isometric embeddings $ \gamma_X \colon X \to Z $ and $ \gamma_Y \colon Y \to Z $ such that
    \begin{equation}
        \dhf^Z(\gamma_X(X), \gamma_Y(Y)) = \e,
    \end{equation}
    where $ \dhf^Z $ denotes the Hausdorff distance respect the distance $ d_Z $. Consider the subspace $ \gamma_X(X) \cup \gamma_Y(Y) \subseteq Z $ with the induced metric from $ Z $. As both $ X $ and $ Y $ are finite, let
    \begin{equation}
        n \coloneq \#(X) + \#(Y). 
    \end{equation}
    Hence, by Lemma \ref{lemma:haus-aux-1}, there exists an isometric embedding
    \begin{equation}
        \gamma \colon (\gamma_X(X) \cup \gamma_Y(Y), d_Z) \to (\R^n, \ell^\infty).
    \end{equation}
    Let $ \dhf^\infty $ denote the Hausdorff distance respect the distance $ d_\infty $. We then have
    \begin{equation}
        \dhf^\infty(\gamma \circ \gamma_X(X), \gamma \circ \gamma_Y(Y)) = \dhf^Z(\gamma_X(X), \gamma_Y(Y)) = \e.
    \end{equation}
    Let $ \delta_X $ be the distance function from a point in $ \R^n $ to $ X $, and analogously, let $ \delta_X $ be the distance function to $ Y $. In $ \ell^\infty $ norm, by hoy we defined $\e$, we have
    \begin{equation}
        \| \delta_X - \delta_Y \|_\infty = \max_{i=1, \dots, n} | \delta_{x_i} - \delta_{y_i} | \leq \e.
    \end{equation}
    As distance functions are linear, both $ \delta_X $ and $ \delta_Y $ are lower envelopes of picewise-linear functions and therefore they are picewise-linear too. Hence,  both $ \delta_X $ and $ \delta_Y $ are tame and continuous so by Theorem \ref{theorem:edelsbrunner-stability} we have
    \begin{equation}
        \db(D(\delta_X), D(\delta_Y)) \leq \| \delta_X - \delta_Y \|_\infty \leq \e.
    \end{equation}

    Let $\alpha \in \R $. Define an off-set of radius $ \alpha $ around the image of the embedded of $ X $ into $ R^n $ as
    \begin{equation}
        \gamma \circ \gamma_X(X) \coloneq \bigcup_{x \in \gamma \circ \gamma_X(X)} B_\alpha^{\ell^\infty}(x),
    \end{equation}
    where $ B_\alpha^{\ell^\infty}(x) $ denotes de ball of radius $ \alpha $ and center $ x $ using distance $ d_\infty $. As balls in $ \ell^\infty $ are hypercubes, they are convex, and therefore their intersection is either empty or contractible. By Lemma \ref{lemma:haus-aux-3} know that $ \delta_X $ has the same persistence diagram as the Čech complex $ \cf(\gamma \circ \gamma_X, \ell^\infty) $. By Lemma \ref{lemma:haus-aux-2}, when using the $ \ell^\infty $-norm, Čech and Rips complexes coincide and so do their filtrations. As $ \gamma \circ \gamma_X $ is an isometric embedding, we then have
    \begin{equation}
        \cf(\gamma \circ \gamma_X, \ell^\infty) = \rf(\gamma \circ \gamma_X, \ell^\infty) = \rf(X, d_X).
    \end{equation}
    Hence, the persistence diagram of $ \rf(X, \ell^\infty) $ is the same as the persistence diagram of $ \gamma_X $. The same is true taking $ Y $ and therefore we have
    \begin{equation}
        \db(D(\rf(X, d_X)), D(\rf(Y, d_Y))) = \db(D(\gamma_X), D(\gamma_Y)) \leq \e.
    \end{equation}
\end{proof}


\begin{proposition}
    Let $ (X, d_X) $, $ (Y, d_Y) $ be finite metric spaces. Then, for any $ k \in \N$, the bottleneck distance
    $$
        \db(D_k(\rf(X, d_X)), D_k(\rf(Y, d_Y))),
    $$
    is a tight lower bound of
    $$
        \dgh((X, d_X), (Y, d_Y)).
    $$
    That is, it is the largest possible lower bound.
\end{proposition}

\begin{theorem}[Theorem 3.2, \cite{chazal}]
    Let $ (X, d_X) $, $ (Y, d_Y) $ be finite metric spaces endowed with the functions $ f \colon X \to \R $ and $ g \colon Y \to \R $. Then
    $$
        \db(D_k(\rf(X, d_X, f)), D_k(\rf(Y, d_Y, g))) \leq \dgh^1((X, d_X, f), (Y, d_Y, g)).
    $$
\end{theorem}