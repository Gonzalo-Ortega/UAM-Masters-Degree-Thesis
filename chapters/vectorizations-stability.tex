
\chapter{Vectorizations' Stability Theorems} \label{chap:vectorizations}

The study of stability in topological data analysis is crucial for ensuring that small perturbations in the input data lead to controlled changes in its topological representations. This chapter explores various vectorization methods—persistence landscapes, persistence images, and Euler curves—and their associated stability theorems. By establishing bounds on how these vectorizations change with respect to distances between persistence modules or diagrams, we gain a deeper understanding of their robustness in practical applications.

Section \ref{sec:landscapes} reviews the results of the 2015 paper from Peter Bubenik \cite{bubenik}. Section \ref{sec:images} follows the 2020 paper Henry Adams et al. \cite{adams} and Section \ref{sec:euler-curves} presents the results from Paweł Dłotko and Davide Gurnari 2023 paper, \cite{dlotko}.

\section{Persistence landscapes} \label{sec:landscapes}

Persistence landscapes provide a functional summary of persistence diagrams, converting them into a sequence of real-valued functions that capture topological features in a more tractable form. This section introduces the definition of persistence landscapes, their key properties, and stability results that relate the landscape distance to the bottleneck and Wasserstein distances between persistence diagrams.

Recall from Definition \ref{def:persistent-homology} that given some persistence module $ (V, \pi) $ and a fixed $k$ the $k$-th persistent Betti number associated to $ x \leq y \in \R $ is given by
\begin{equation}
    \beta_x^y(V) = \dim (\im \pi_{x \leq y}).
\end{equation}

\begin{lemma}[Lemma 1, \cite{bubenik}] \label{lemma:landscapes-aux-1}
    Let $ (V, \pi) $ be a persistence module and let $ a \leq b \leq c \leq d \in \R $. Then $ \beta_b^c \geq \beta_a^d $.
\end{lemma}
\begin{proof}
    Since $ \pi_{a \leq d} = \pi_{c \leq d} \circ \pi_{b \leq c} \circ \pi_{a \leq b} $, the dimension of the image of $ \pi_{a \leq d} $ must be smaller than the one of just $ \pi_{b \leq c} $.
\end{proof}

\begin{lemma}[Lemma 2, \cite{bubenik}] \label{lemma:landscapes-aux-2}
    Let $ 0 \leq h_1 \leq h_2 $ be real numbers. Then $ \beta_{t-h_1}^{t+h_1} \geq \beta_{t-h_2}^{t+h_2} $.
\end{lemma}
\begin{proof}
    ...
\end{proof}

Let $ \overline{\R} \coloneq \R \cup \{-\infty, \infty\} $ denote the extended real numbers. 

\begin{definition}[Persistence landscape] \label{def:persistence-landscape}
    A {\bf persistence landscape} is a function $ \lambda \colon \N \times \R \to \overline \R $, defined as
    \begin{equation}
        \lambda(k, t) \coloneq \sup \{ m \geq 0 \mid \beta^{t-m, t+m} \geq k\}.
    \end{equation}
    Note that this function can also be seen as a sequence of function $ \lambda_k \colon \R \to \overline{\R} $, where $ \lambda_k(t) = \lambda(k, t) $.
\end{definition}

\begin{definition}[$K$-Lipschitz]
    Let $ (X, d_X) $, $ (Y, d_Y) $ be two metric spaces and let $ K > 0 $. A {\bf $K$-Lipschitz} map is a map $ f \colon (X, d_X) \to (Y, d_Y) $, such that for every $ x_1, x_2 \in X $, 
    \begin{equation}
        d_Y(f(x_1), f(x_2)) \leq K d_X(x_1, x_2).
    \end{equation}
\end{definition}

\begin{lemma}[Lemma 3, \cite{bubenik}] \label{lemma:landscapes-aux-3}
    Let $ \lambda_k \colon \R \to \overline{\R} $ be an element of a persistence landscape. The following properties are verified.
    \begin{enumerate}
        \item $\lambda_k(t) \geq 0$,
        \item $\lambda_k(t) \geq \lambda_{k+1}(t)$,
        \item $\lambda_k$ is $1$-Lipschitz, that is, for $ t, s \in \R $, $ | \lambda_k(t) - \lambda_k(s) | \leq |t-s| $.
    \end{enumerate}
\end{lemma}
\begin{proof}
    Properties 1. and 2. came directly from the definition. For 3., suppose $ \lambda_k(s) \leq \lambda_k(t) $. If $ \lambda_k(t) \leq |t -s | $ then, of course, $ \lambda_k(t) - \lambda_k(s) \leq \lambda_k(t) \leq |t-s| $. Else, if $ \lambda_k(t) >  |t - s | $, we can take some $ h \in (0, \lambda_k(t) -  |t - s |) $ verifying
    \begin{equation}
        t - \lambda_k(t) < s- h < s + h < t + \lambda_k(t).
    \end{equation}
    Hence, by Lemma \ref{lemma:landscapes-aux-1}
\end{proof}

\begin{definition}[$p$-landscape distance]
    Let $ W $ and $ W $ be two persistence modules, and let $ \lambda $ and $ \lambda' $ its corresponding persistence landscapes. Let $ 1 \leq p \leq \infty $. The {\bf $p$-landscape distance between persistence modules} $ V $ and $ W $ is defined as
    \begin{equation}
        \Lambda_p(V, W) \coloneq \|\lambda - \lambda' \|_p.
    \end{equation}
    Similarly, if $ D $ and $ D' $ are two persistence diagrams, and are $ \lambda $ and $ \lambda' $ its corresponding persistence landscapes, The {\bf $p$-landscape distance between persistence diagrams} $ V $ and $ W $ is defined as
    \begin{equation}
        \Lambda_p(D, D') \coloneq \|\lambda - \lambda' \|_p.
    \end{equation}
\end{definition}

\begin{theorem}[Theorem 17, \cite{bubenik}] \label{theorem:landscapes-aux}
    Let $ (V, \pi) $ and $ (W, \theta) $ two persistence modules. Then their $\infty$-landscape distance between them is a lower bound of their interleaving distance. That is
    \begin{equation}
        \Lambda_{\infty}(V, W) \leq \di(V, W)
    \end{equation}
\end{theorem}
\begin{proof}
    Suppose $ V $ and $ W $ are $\delta$-interleaved, recall definition \ref{def:interleaved-modules}. Then, considering $ \pi_{t-m \leq t+m} $ and $ \pi_{t-m + \delta \leq t+m - \delta } $, we have that $ t-m \leq t-m + \delta \leq t+m - \delta \leq t+m $, hence by Lemma \ref{lemma:landscapes-aux-1} we have
    \begin{equation}
        \beta_{t-m+\delta}^{t+m-\delta}(W) \geq \beta_{t-m}^{t+m}(V).
    \end{equation}
    Now if $ \delta $ and $ \delta' $ are the corresponding persistence landscapes of $ V $ and $ W $, by Definition \ref{def:persistence-landscape}, for every $ k \leq 1 $, $ \lambda'(k, t) \geq \lambda(k, t) - \delta $. Hence, it follows
    \begin{equation}
        \|\lambda - \lambda'\|_\infty \leq \delta.
    \end{equation}
\end{proof}

\begin{theorem}[Theorem 12, \cite{bubenik}]
    Consider the persistence modules $ V = F_x $, $ W = G_x $ given by the maps $ f, g \colon X \to \R $. Then
    \begin{equation}
        \Lambda_\infty(V, W) \leq \| f - g\|_\infty
    \end{equation}
\end{theorem}
\begin{proof}
    The proofs follows from applying Theorem \ref{theorem:landscapes-aux}, Theorem \ref{chap:interleaving-stability} and Theorem \ref{theorem:edelsbrunner-stability}, in such order. Hence
    \begin{equation}
        \Lambda_\infty(V, W) \leq \di(V, W) = \db(D(f), D(g)) \leq \| f - g\|_\infty.
    \end{equation}
\end{proof}

\begin{theorem}[Theorem 13, \cite{bubenik}]
    Let $ D $ and $ D' $ be two persistence diagrams, then
    \begin{equation}
        \Lambda_\infty(D, D') \leq \db(D, D').
    \end{equation}
\end{theorem}
\begin{proof}
    ...
\end{proof}

\section{Persistence images} \label{sec:images}

Persistence images offer an alternative vectorization by transforming persistence diagrams into finite-dimensional representations through kernel density estimation. Here, we define persistence surfaces and images, analyze their stability under the 1-Wasserstein distance, and derive bounds that ensure their reliability in machine learning and statistical applications.

\begin{definition}[Persistence surface]
    Let $ D $ be a persistence diagram. Let $ T \colon \R^2 \to \R^2 $ be the linear transformation $ T(x, y) = (x, y-x) $. Fix a nonnegative weighting function $ f \colon \R^2 \to R $ that is zero along the horizontal axis, continuous and picewise differentiable. Fix a differentiable probability distribution $ \phi_u \colon \R^2 \to \R $, with mean $ u \in \R^2 $. The {\bf persistence surface} associated to $ D $, by $ f $ and $ \phi_u $ is a function $ \rho_D \colon \R^2 \to \R $ defined as
    \begin{equation}
        \rho_D(z) \coloneq \sum_{u \in T(D)} f(u) \phi_u(z).
    \end{equation}
\end{definition}

\begin{definition}[Persistence image]
    Let $ D $ be a persistence diagram with an associated persistence surface $ \rho_D $. The {\bf persistence image} of $ D $ by $ \rho_D $ is the collection $ \rho $ of {\bf pixels}
    \begin{equation}
        I(\rho_D)_p \coloneq \iint_p \rho_B dy dx.
    \end{equation}
\end{definition}

Let $ h \colon \R^2 \to \R $ be a differentiable function. We will denote the maximal norm of the gradient vector of $ h $ as
\begin{equation}
    |\nabla h | = \sup_{z \in \R^2} \| \nabla h(z) \|_2.
\end{equation}
Hence, by the fundamental theorem of calculus for line integrals, for every $ u, v \in \R^2 $, we have
\begin{equation}
    |h(u) - h(v)| \leq |\nabla h| \|u-v\|_2.
\end{equation}
Now, for a differentiable probability distribution $ \phi_u \colon \R^2 \to \R $, with mean $ u \in \R^2 $ we can denote $ |\nabla \phi_u| $ as $ |\nabla \phi| $ and $ \|\phi_u\| $ as $ \|\phi\| $ because both the maximal directional derivative and the uniform norm of a fixed differentiable probability function are invariable under translation.

\begin{lemma}[Lemma 3, \cite{adams}] \label{lemma:images-aux}
    Let $ u, v \in \R^2 $. Fix a nonnegative weighting function $ f \colon \R^2 \to R $ that is zero along the horizontal axis, continuous and picewise differentiable. Fix a differentiable probability distribution $ \phi_u \colon \R^2 \to \R $, with mean $ u \in \R^2 $. The following inequality asserts.
    \begin{equation}
        \|f(u) \phi_u - f(v)\phi_v \|_\infty \leq (\|f\|_\infty |\nabla \phi| + \|\phi\|_\infty |\nabla f |) \| u - v \|_2.
    \end{equation}
\end{lemma}
\begin{proof}
    For any $ z \in \R^2 $ we have
    \begin{equation}
        | \phi_u(z) - \phi_v(z) | = | \phi_u(z) - \phi_u(z + u - v) | \leq |\nabla \phi | \|u - v \|_2.
    \end{equation}
    Hence,
    \begin{equation}
        \| \phi_u - \phi_v \| \leq |\nabla \phi | \|u - v \|_2.
    \end{equation}
    Applying this, we can now develop    
    \begin{align}
        |f(u)\phi_u(z) - f(v)\phi_v(z)| &= |f(u)(\phi_u(z) - \phi_v(z)) + (f(u) - f(v))\phi_v(z)| \\
        &\leq \|f\|_\infty |\phi_u(z) - \phi_v(z)| + \|\phi\|_\infty |f(u) - f(v)| \\
        &\leq \|f\|_\infty \|\nabla \phi\| \|u - v\|_2 + \|\phi\|_\infty \|\nabla f\| \|u - v\|_2 \\
        &= (\|f\|_\infty \|\nabla \phi\| + \|\phi\|_\infty \|\nabla f\|)\|u - v\|_2.
    \end{align}
\end{proof}

Persistence surfaces are stable with respect to the $1$-Wasserstein distance (see \ref{def:Wasserstein}).

\begin{theorem}[Theorem 4, \cite{adams}] \label{theorem:images-1}
    Let $ D, D' $ be two persistent diagrams and $ \rho_D, \rho_{D'} $ two  persistence surfaces associated to each diagram respectively. Then
    \begin{equation}
        \| \rho_B - \rho_{B'} \|_\infty \leq \sqrt{10} (\|f\|_\infty |\nabla \phi | + \|\phi\|_\infty |\nabla f |) \omega_1(D, D').
    \end{equation}
\end{theorem}
\begin{proof}
    Since $D$ and $D'$ consist of finitely many points, there exists a matching $\gamma$ that achieves the infimum in the Wasserstein distance. Then
    \begin{align}
        \|\rho_B - \rho_{B'}\|_{\infty} 
        &= \left\| \sum_{u \in T(B)} f(u)\phi_u - \sum_{u \in T(B)} f(\gamma(u))\phi_{\gamma(u)} \right\|_{\infty} \\
        &\leq \sum_{u \in T(B)} \left\| f(u)\phi_u - f(\gamma(u))\phi_{\gamma(u)} \right\|_{\infty},
    \end{align}
    We now apply Lemma \ref{lemma:images-aux} and then note that $ \|\cdot \|_2 \leq \sqrt{2} \| \cdot \| in \R^2 $. and that $ \|T(\cdot) \|_2 \leq \sqrt{5} \| \cdot \| $,
    \begin{align}
        \sum_{u \in T(B)} \left\| f(u)\phi_u - f(\gamma(u))\phi_{\gamma(u)} \right\|_{\infty}
        &\leq \sqrt{2} (\|f\|_{\infty} \|\nabla \phi\| + \|\phi\|_{\infty} \|\nabla f\|) 
        \sum_{u \in T(B)} \|u - \gamma(u)\|_{\infty} \\
        &\leq \sqrt{10} (\|f\|_{\infty} \|\nabla \phi\| + \|\phi\|_{\infty} \|\nabla f\|) 
        \sum_{u \in B} \|u - \gamma(u)\|_{\infty} \\
        &= \sqrt{10} (\|f\|_{\infty} \|\nabla \phi\| + \|\phi\|_{\infty} \|\nabla f\|) W_1(B, B').
    \end{align}
\end{proof}

Persistence images are stable with respect to the $1$-Wasserstein distance.

\begin{theorem}[Theorem 5, \cite{adams}] \
    Let $ A $ be be the maximum area of any pixel in the image, $ A' $ the total area of the image, and $ n $ the number of pixels in the image. Then
    \begin{align}
        &\| I(\rho_B) - I(\rho_{B'}) \|_\infty \leq \sqrt{10} A (\|f\|_\infty |\nabla \phi | + \|\phi\|_\infty |\nabla f |) \omega_1(D, D'), \\
        &\| I(\rho_B) - I(\rho_{B'}) \|_1 \leq \sqrt{10} A' (\|f\|_\infty |\nabla \phi | + \|\phi\|_\infty |\nabla f |) \omega_1(D, D'), \\
        &\| I(\rho_B) - I(\rho_{B'}) \|_2 \leq \sqrt{10n} A (\|f\|_\infty |\nabla \phi | + \|\phi\|_\infty |\nabla f |) \omega_1(D, D'). \\
    \end{align}
\end{theorem}
\begin{proof}
    Note for any pixel $ p $ with area $ A(p) $ we have
    \begin{align}
        |I(\rho_{B})_{p} - I(\rho_{B'})_{p}| &= \left| \iint_{p} \rho_{B} \, dy dz - \iint_{p} \rho_{B'} \, dy dx \right| \\
        &= \left| \iint_{p} \rho_{B} - \rho_{B'} \, dy dx \right| \\
        &\leq A(p) \|\rho_{B} - \rho_{B'}\|_{\infty} \\
        &\leq \sqrt{10} A(p) \big( \|f\|_{\infty} |\nabla \phi| + \|\phi\|_{\infty} |\nabla f| \big) \omega_{1}(B, B'),
    \end{align}
    where the last inequality comes by applying Theorem \ref{theorem:images-1}. Hence we have
    \begin{align}
        \|I(\rho_{B}) - I(\rho_{B^{\prime}})\|_{\infty} &\leq \sqrt{10}A(\|f\|_{\infty}|\nabla\phi| + \|\phi\|_{\infty}|\nabla f|\big)\omega_{1}(B,B^{\prime}) \\
        \|I(\rho_{B}) - I(\rho_{B^{\prime}})\|_{1} &\leq \sqrt{10}A^{\prime}\big(\|f\|_{\infty}|\nabla\phi| + \|\phi\|_{\infty}|\nabla f|\big)\omega_{1}(B,B^{\prime}) \\
        \|I(\rho_{B}) - I(\rho_{B^{\prime}})\|_{2} &\leq \sqrt{n}\|I(\rho_{B}) - I(\rho_{B^{\prime}})\|_{\infty} \\
        &\leq \sqrt{10n}A\big(\|f\|_{\infty}|\nabla\phi| + \|\phi\|_{\infty}|\nabla f|\big)\omega_{1}(B,B^{\prime}).
    \end{align}
\end{proof}

\section{Euler curves}  \label{sec:euler-curves}

The Euler characteristic curve provides a concise topological signature by tracking the evolution of the Euler characteristic across a filtration. This section examines its definition, connection to persistence diagrams, and stability properties, demonstrating how it serves as a stable summary for comparing filtered cell complexes.

\begin{definition}[Euler characteristic]
    Let $ K $ be a simplicial complex, and let $ K^p $ be its $p$-skeleton. The {\bf Euler characteristic} of $ K $ is the alternating sum of the number of cells in its dimension
    \begin{equation}
        \chi(K) \coloneq \sum_d (-1)^d \#(K^d).
    \end{equation}
\end{definition}

\begin{definition}
    Let $ K $ be a simplicial complex. Let $ f \colon K \to \R $ be a filtration function. The {\bf Euler characteristic curve} is a function that assign an Euler characteristic $ \chi $ for each filtration level $ t \in \R $. 
    \begin{equation}
        \ecc(K, t) \coloneq \chi(K_t),
    \end{equation}
    where $ K_t f^{-1} (-\infty, t] $.
\end{definition}

\begin{proposition}[Proposition 2, \cite{dlotko}]
    Let $ X, Y $ be two filtered cell complexes and let $ D(X), D(Y) $ be its respective persistence diagrams. Then,
    \begin{equation}
        \| \ecc(X, t) - \ecc(Y, t) \|_1 \leq \sum_k 2 \omega_1(D(X), D(Y)).
    \end{equation}
\end{proposition}
\begin{proof}
    ...
\end{proof}
