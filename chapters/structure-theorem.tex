\chapter{Structure Theorem} \label{chap:structure}
The Structure Theorem for persistence modules is referred to as the ``first miracle'' of persistence homology \cite{nanda}. This algebraic property allows to express a persistence module of finite type as a direct sum of finitely many interval modules. This enables to express a persistence module by the bars in a barcode, which latter allows to compare bottleneck distance between the barcodes, with the interleaving distance between the modules.

The proof of the theorem requires the algebraic structure theorem for finitely generated modules over a principal domain. Section \ref{sec:structure-algebraic} proofs the algebraic structure theorem for finitely generated modules over a principal ideal domain. Section \ref{sec:structure-persistence} then gives a proof of the Structure Theorem for persistence diagrams.

\section{Algebraic structure theorem} \label{sec:structure-algebraic}

\section{Graded modules} \label{sec:preliminaries-persistence-modules}

\begin{definition}[Graded ring]
    Let $ R $ be a ring. It is said that $ R $ is a {\bf graded ring} if it can be decomposed into a direct sum of additive groups
    \begin{equation}
        R = \bigoplus_{n=1}^{\infty} R_n = R_1 \oplus R_2 \oplus \dots
    \end{equation}
    such that for all $ n, m \geq 0 $, 
    \begin{equation}
        R_n \cdot R_m \subseteq R_{n+m}.
    \end{equation}
\end{definition}

\begin{example}[Example 4.4, \cite{wang}] \label{ex:graded-ring}
    Let $ \F $ be a field, the polynomial ring over it, $ \F[x] $, is a graded ring. It can be decomposed as
    \begin{equation}
        \F[x] = \bigoplus_{i=0}^\infty x^i \cdot \F = \bigoplus_{i=0}^\infty \{c x^i \mid c \in \F\}.
    \end{equation}
    Also, polynomial multiplication verifies the degree of the product of two monomials is the sum of the degrees of the factors.
\end{example}

\begin{definition}[Graded ideal]
    Let $ R $ be a graded ring. A {\bf graded ideal} is a two sided ideal $ I \subseteq R $ that can be decomposed into a direct sum
    \begin{equation}
        I = \bigoplus_{n=0}^{\infty} I_n
    \end{equation}
    where for each $n \geq 0 $, $ I_n = I \cap R_n $.
\end{definition}

\begin{example}
    Note that in Example \ref{ex:graded-ring}, every ideal is a graded ideal. No see an example of non graded ideals take the two-dimensional polynomial ring over a field, $ R = \F[x, y] $. There are also many graded ideals in $ R $, take, for example, the ideal generated by the two dimensional monomials $ I = \langle x^2, xy \rangle $.

    Take now an ideal generated by non homogeneous elements as $ I' = \langle x^2 + y \rangle $. Then it happens that $ x^2 \in I'_2 = I' \cap R^2 $ and $ y \in I'_1 = I' \cap R^1 $, but either $ x^2 \in I' $ nor $ y \in I' $.
\end{example}

\begin{definition}[Graded module, Definition 4.7 \cite{wang}]
    Let $M$ be a left module over a graded ring $ R $. It is said that $ M $ is a {\bf left graded module} if it can be decomposed into a direct sum
    $$
        M = \bigoplus_{n=1}^{\infty} M_n
    $$
    if for each $n, m \geq 0 $, $ R_n M_m \subseteq M_{n+m} $.
\end{definition}

\begin{theorem}[Chapter IV, Theorem 6.12, \cite{hungerford}] \label{theorem:structure}
    Let $ M $ be a  finitely generated module over a principal ideal domain $R$. There exist a finite sequence of proper ideals $ (d_1) \supseteq (d_2) \supseteq \dots \supseteq (d_n) $ such that
    $$
        M \cong \bigoplus_{i=1}^n R / (d_i).
    $$
\end{theorem}

Due to the lengthy concepts needed to prove it, we will refer to it as a well known fact. An introduction to module theory and a detailed proof of the theorem of Fact \ref{fact:structure} can be found at \cite[Chapter IV]{hungerford}.

\section{Structure theorem for persistence diagrams} \label{sec:structure-persistence}
In addition to Theorem \ref{theorem:structure}, we will use the following simple algebraic statement.

\begin{proposition}[Proposition 4.6, \cite{wang}] \label{prop:graded-iff-homo}
    An ideal $ I \subseteq R $ is graded if and only if it is generated by homogeneous elements.
\end{proposition}
\begin{proof}
    First, if $ I $ is a graded ideal $ I = \bigoplus_p I^p $ and is generated by $ \bigcup_p I^p $. Then, each
    $$
        I^p = I \cap R^p \subseteq R^p 
    $$
    is a subset of homogeneous elements. Therefore, $ I $ is generated by homogeneous elements.
    
    Now, let $ I $ be generated by a set $ X $ of homogeneous elements. For sure, $ I \cap R^p \subseteq I $, so we just need to prove the converse inclusion. As $ I $ is generated by $ X $, its elements $ u \in I $ are of the form
    \begin{align}
        u = \sum_i r_i x_i s_i, \label{eq:structure-u1}
    \end{align}
    for $ r_i, s_i \in R $ and $ x_i \in X $. And as $ I \subseteq R $, also,
    \begin{align}
        u = \sum_p u_p,
    \end{align}
    for $ u_p \in R^p $. For every term in \eqref{eq:structure-u1}, we have
    \begin{align}
        r_i = \sum_j r_{i,j}, &  & s_i = \sum_l s_{i,l}, 
    \end{align}
    with each $ r_{i,j} $, $ s_{i,l} $ being homogeneous. Therefore, combining all we have that
    \begin{align}
        u = \sum_i \sum_{j, l} r_{i, j} x_i s_{i, l}. \label{eq:structure-u-total}
    \end{align}
    Each term in \eqref{eq:structure-u-total} is homogeneous as is a product of homogeneous elements. Thus $ u_p $ is the sum of those terms, and $ u $ has degree $ p $. Therefore $ u_p \in I $ and $ I \subseteq I \cap R^p $.
\end{proof}

\begin{theorem}[Proposition 4.8, \cite{wang}]\label{theorem:persistence-structure}
    Let $ (V, \pi) $ be a persistence module. There exist a barcode $ \barc(V, \pi) $, with $ \mu \colon \barc (V, \pi) \longrightarrow \mathbb N $, the multiplicity of the barcode intervals, such as there is a unique direct sum decomposition
    \begin{align}
        V \cong \bigoplus_{I \in  \barc (V)} \mathbb F (I)^{\mu (I)}. \label{eq:structure}
    \end{align}
\end{theorem}
\begin{proof}
    $ V $ is of finite type, so it is a finite $ \mathbb F[x] $-module. As $ \mathbb F $ is a field, $ \mathbb F[x] $ is a principal ideal domain, therefore, $ V $ is a finitely generated module over a principal ideal domain. Using Fact \ref{theorem:structure}, $ V $ can be decompose in the direct sum of its free and torsion subgroups, $ F \oplus T $. Thus, we have
    \begin{align}
        F &= \bigoplus_{i\geq q} x^i \cdot \mathbb F \\
        T &= \bigoplus_{i\geq q} R^i / I^i.
    \end{align}
    Each $ x^i \cdot \mathbb F $ is isomorphic to ideals of the form $ (x^q) $. By Proposition \ref{prop:graded-iff-homo}, each $ R^i / I^i $ is isomorphic to some quotient of graded ideals of the form $ (x^p) / (x^r)$. Note that the free subgroup can be seen as a particular case of the torsion group taking $ r = 0 $. Thus $ V $ can be decompose as described in \eqref{eq:structure}.
\end{proof}
