\chapter{Structure Theorem}
\section{Structure theorem for finitely generated modules over a principal ideal domain}

\begin{theorem}[Chapter IV, Theorem 6.12, \cite{hungerford}] \label{theorem:structure}
    Let $ M $ be a  finitely generated module over a principal ideal domain $R$. There exist a finite sequence of proper ideals $ (d_1) \supseteq (d_2) \supseteq \dots \supseteq (d_n) $ such that
    $$
        M \cong \bigoplus_{i=1}^n R / (d_i).
    $$
\end{theorem}

\section{Structure theorem for persistence diagrams}
The Structure Theorem for persistence modules is referred to as the ``first miracle'' of persistence homology \cite{nanda}. This algebraic property allows to express a persistence module of finite type as a direct sum of finitely many interval modules. Its proof requires the algebraic structure theorem for finitely generated modules over a principal domain.

%Due to the lengthy concepts needed to prove it, we will refer to it as a well known fact. An introduction to module theory and a detailed proof of the theorem of Fact \ref{fact:structure} can be found at \cite[Chapter IV]{hungerford}.

In addition to Theorem \ref{theorem:structure}, we will use the following simple algebraic statement.

\begin{proposition}[Proposition 4.6, \cite{wang}] \label{prop:graded-iff-homo}
    An ideal $ I \subseteq R $ is graded if and only if it is generated by homogeneous elements.
\end{proposition}
\begin{proof}
    First, if $ I $ is a graded ideal $ I = \bigoplus_p I^p $ and is generated by $ \bigcup_p I^p $. Then, each
    $$
        I^p = I \cap R^p \subseteq R^p 
    $$
    is a subset of homogeneous elements. Therefore, $ I $ is generated by homogeneous elements.
    
    Now, let $ I $ be generated by a set $ X $ of homogeneous elements. For sure, $ I \cap R^p \subseteq I $, so we just need to prove the converse inclusion. As $ I $ is generated by $ X $, its elements $ u \in I $ are of the form
    \begin{align}
        u = \sum_i r_i x_i s_i, \label{eq:structure-u1}
    \end{align}
    for $ r_i, s_i \in R $ and $ x_i \in X $. And as $ I \subseteq R $, also,
    \begin{align}
        u = \sum_p u_p,
    \end{align}
    for $ u_p \in R^p $. For every term in \eqref{eq:structure-u1}, we have
    \begin{align}
        r_i = \sum_j r_{i,j}, &  & s_i = \sum_l s_{i,l}, 
    \end{align}
    with each $ r_{i,j} $, $ s_{i,l} $ being homogeneous. Therefore, combining all we have that
    \begin{align}
        u = \sum_i \sum_{j, l} r_{i, j} x_i s_{i, l}. \label{eq:structure-u-total}
    \end{align}
    Each term in \eqref{eq:structure-u-total} is homogeneous as is a product of homogeneous elements. Thus $ u_p $ is the sum of those terms, and $ u $ has degree $ p $. Therefore $ u_p \in I $ and $ I \subseteq I \cap R^p $.
\end{proof}

\begin{theorem}[Proposition 4.8, \cite{wang}]\label{theorem:persistence-structure}
    Let $ (V, \pi) $ be a persistence module. There exist a barcode $ \barc(V, \pi) $, with $ \mu \colon \barc (V, \pi) \longrightarrow \mathbb N $, the multiplicity of the barcode intervals, such as there is a unique direct sum decomposition
    \begin{align}
        V \cong \bigoplus_{I \in  \barc (V)} \mathbb F (I)^{\mu (I)}. \label{eq:structure}
    \end{align}
\end{theorem}
\begin{proof}
    $ V $ is of finite type, so it is a finite $ \mathbb F[x] $-module. As $ \mathbb F $ is a field, $ \mathbb F[x] $ is a principal ideal domain, therefore, $ V $ is a finitely generated module over a principal ideal domain. Using Fact \ref{theorem:structure}, $ V $ can be decompose in the direct sum of its free and torsion subgroups, $ F \oplus T $. Thus, we have
    \begin{align}
        F &= \bigoplus_{i\geq q} x^i \cdot \mathbb F \\
        T &= \bigoplus_{i\geq q} R^i / I^i.
    \end{align}
    Each $ x^i \cdot \mathbb F $ is isomorphic to ideals of the form $ (x^q) $. By Proposition \ref{prop:graded-iff-homo}, each $ R^i / I^i $ is isomorphic to some quotient of graded ideals of the form $ (x^p) / (x^r)$. Note that the free subgroup can be seen as a particular case of the torsion group taking $ r = 0 $. Thus $ V $ can be decompose as described in \eqref{eq:structure}.
\end{proof}
