\markboth{Abstract}{Abstract}

\subsection*{Abstract}
This thesis presents a unified treatment of structure and stability theorems in Topological Data Analysis (TDA), focusing on persistent homology. The Structure Theorem establishes that persistence modules decompose into interval modules, enabling representation via barcodes or persistence diagrams. Four stability pillars are rigorously developed: Interleaving Stability proves an isometry between the algebraic interleaving distance for persistence modules and the combinatorial bottleneck distance for barcodes; Hausdorff Stability bounds the bottleneck distance between persistence diagrams of functions by their $L^\infty$-distance; Gromov-Hausdorff Stability extends this to metric spaces, linking the bottleneck distance of Vietoris-Rips or Čech filtrations to the Gromov-Hausdorff distance between spaces; Vectorization Stability guarantees robustness for practical summaries like persistence landscapes, images, and Euler curves, enabling integration with machine learning. By synthesizing tools from algebraic topology, category theory, and metric geometry, this work fortifies the mathematical foundations of TDA and facilitates reliable applications in scientific domains and shape quantification in high-dimensional data.

\subsection*{Resumen}
Esta tesis presenta un tratamiento unificado de teoremas de estructura y estabilidad en el Análisis Topológico de Datos (TDA), centrándose en la homología persistente. El Teorema de Estructura establece que los módulos de persistencia se descomponen en módulos de intervalo, permitiendo su representación mediante códigos de barras o diagramas de persistencia. Se desarrollan rigurosamente cuatro pilares de estabilidad: la Estabilidad de Entrelazado demuestra una isometría entre la distancia de entrelazado algebraico para módulos de persistencia y la distancia de cuello de botella para códigos de barras; la Estabilidad de Hausdorff acota la distancia bottleneck entre diagramas de persistencia de funciones por su norma $ L^\infty $; la Estabilidad de Gromov-Hausdorff extiende esto a espacios métricos, vinculando la distancia bottleneck de filtraciones de Vietoris-Rips o Čech con la distancia de Gromov-Hausdorff entre espacios; la Estabilidad de Vectorizaciones garantiza robustez para resúmenes prácticos como paisajes de persistencia, imágenes de persistencia y curvas de Euler, facilitando su integración con aprendizaje automático. Al sintetizar herramientas de topología algebraica, teoría de categorías y geometría métrica, este trabajo fortalece los fundamentos matemáticos del TDA y posibilita aplicaciones confiables en dominios científicos y la cuantificación de formas en datos de alta dimensión.


\subsection*{Key words}
Topological Data Analysis, Persistent homology, Topological summary, Stability.
