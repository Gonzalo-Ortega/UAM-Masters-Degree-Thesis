\section{Preliminaries}
(TO DO) The contents of this chapter are based on \cite{nanda}, \cite{polterovich} and \cite{wang}.

\begin{definition}[Graded ring]
    
\end{definition}

\begin{definition}[Graded ideal]
    
\end{definition}

\begin{definition}[Graded moudule]
    
\end{definition}

\begin{definition}[Persistance module, finite type]
%    $ (V, \pi) $, where $ V = \{V_t\}_{t \in \mathbb R} $ is a collection of finite dimensional vector spaces over a field $ \mathbb F $, and $ \pi = \{ \pi_{s \leq t} \} $ is a collection of linear maps $ \pi_{s \leq t}\colon V_s \rightarrow V_t $.
\end{definition}

\begin{definition}[Module morphism, shift]
    
\end{definition}

\begin{definition}[Interval module]
    
\end{definition}

\begin{definition}[Direct sum of persistance modules]
    
\end{definition}

\begin{definition}[Barcode]

\end{definition}

\begin{definition}[$\delta$-interleaving moudules]
    
\end{definition}

\begin{definition}[Interleaving distance]
    
\end{definition}

\begin{definition}[Multiset matching]
     
\end{definition}

\begin{definition}[$\delta$-matching barcodes] \label{delta-matching}
    
\end{definition}

\begin{definition}[Bottleneck distance]
    
\end{definition}

\newpage
\section{Structure Theorem}
The Structure Theorem for persistence modules 

\begin{theorem}[Structure theorem for finitely generated modules over a principal ideal domain] \label{structure-pure}
    Let $ M $ be a  finitely generated module over a principal ideal domain. There exist a finite sequence of proper ideals $ (d_1) \supseteq (d_2) \supseteq \dots \supseteq (d_n) $ such that
    $$
        M \cong \bigoplus_{i=1}^n R / (d_i).
    $$
\end{theorem}
\begin{proof}
    (TO DO)
\end{proof}

\begin{proposition} \cite[Proposition 4.6]{wang}
    An ideal $ I \subseteq R $ is graded if and only if it is generated by homogeneous elements.
\end{proposition}
\begin{proof}
    (TO DO)
\end{proof}

\begin{theorem}[Structure] \cite[Proposition 4.8]{wang} \label{structure}
    Let $ (V, \pi) $ be a persistence module. There exist a barcode $ \barc(V, \pi) $, with $ \mu \colon \barc (V, \pi) \longrightarrow \mathbb N $, the multiplicity of the barcode intervals, such as there is a unique direct sum decomposition
    $$
        V \cong \bigoplus_{I \in  \barc (V)} \mathbb F (I)^{\mu (I)}.
    $$
\end{theorem}
\begin{proof}
    (INCOMPLETE)
    $ V $ is of finite type, so it is a finite $ \mathbb F[x] $-module. As $ \mathbb F $ is a field, $ \mathbb F[x] $ is a principal ideal domain, therefore, $ V $ is a finitely generated module over a principal ideal domain. Using Fact \ref{structure-pure} $ V $ can be decompose in the direct sum of its free and torsion subgroups, $ F \oplus T $. Thus, we have
    \begin{align*}
        F &= \\
        T &= .
    \end{align*}
    
\end{proof}

\newpage
\section{Stability Theorem}
In this section we are going to give a detailed proof of the first stability theorem for persistence homology. This theorem, the 'geometry miracle' as it is refer to in \cite{nanda}, claims that given two persistence modules, the distance between them, using the interleaving distance, is the same as the distance between their barcodes, using the bottleneck distance. 

For the presented proof we have followed the proceedings of \cite{polterovich}. Hence, we will divide the proof into proving the two inequalities separately. This implies checking that if there exists a $\delta$-matching between two given barcodes, then there exists a $\delta$-interleaving morphism between them. And checking that the other way around also verifies, \ref{prop:interleaving-if-matching}. That is, if there exists a $\delta$-interleaving morphism between two persistence modules, then there exists a $\delta$-matching between their barcodes, \ref{prop:matching-if-interleaving}.

The first claim can be deduced from the Structure Theorem in a rather direct way, proving first the case where our modules are just interval modules.

\begin{lemma} \cite[Exercise 2.2.7]{polterovich} \label{interval-interleaving-if-matching}
    Let $ I, J $ be two $\delta$-matched intervals. Then, their corresponding interval modules $ (\mathbb F (I), \pi) $ and $( \mathbb F (J), \theta) $ are $\delta$-interleaved.
\end{lemma}
\begin{proof}
    Let $ I = (a, b] $, $ J = (c, d] $. If $\rho$ is the $\delta$-matching between them, then $ \rho(I) = J $ and, following Definition \ref{delta-matching}, $ (a, b] \subseteq (c-\delta, d+\delta] $ and $ (c, d] \subseteq (a-\delta, b+\delta] $, with $ b - a > 2\delta $ and $ d - c > 2\delta $. Then, the morphisms
    \begin{equation*}
        \begin{aligned}[c]
        \phi_\delta\colon \mathbb F(I) &\to \mathbb F(J)_\delta\\
        \phi_\delta(\mathbb F(I)_t) &\mapsto \mathbb F(J)_{(t+\delta)}\\
        \end{aligned}
        \quad \text{and} \quad
        \begin{aligned}[c]
        \psi_\delta\colon \mathbb F(J) &\to \mathbb F(I)_\delta\\
        \psi_\delta(\mathbb F(J)_t) &\mapsto \mathbb F(I)_{(t+\delta)}\\
        \end{aligned}
    \end{equation*}
    are well defined as for any $ t \in (a,b], \ t + \delta \in (c, d] $ as $ a + \delta > c $ and $ b + \delta \leq d $. In the same way, for any $ t \in (c,d], \ t + \delta \in (a, b] $. Thus, $ \psi_\delta \circ \phi_\delta (\mathbb F(I)_t) = \psi_\delta(\mathbb F(J)_{(t+\delta)}) = \mathbb F(I)_{(t+2\delta)} = \pi_{t \leq t+2\delta}(\mathbb F(I)_t)$ and $ \phi_\delta \circ \psi_\delta (\mathbb F(J)_t) = \phi_\delta(\mathbb F(I)_{(t+\delta)}) = \mathbb F(J)_{(t+2\delta)} = \theta_{t \leq t+2\delta}(\mathbb F(J)_t)$. Therefore, $ \phi_\delta $ and $ \psi_\delta $ are a pair of $\delta$-interleaving morphisms.
\end{proof}

\begin{proposition} \cite[Theorem 3.0.1]{polterovich} \label{prop:interleaving-if-matching} 
    Given two persistence modules $ V $, $ W $, if there is a $ \delta$-matching between their barcodes, then there is a $ \delta$-interleaving morphism between them.
\end{proposition}
\begin{proof}
    Suppose that $\rho\colon \barc (V) \to \barc (W) $ is a $\delta$-matching between the barcodes of $ V $ and $ W $. By the Structure Theorem \ref{structure}, $ V $ and $ W $ decompose in a finite direct sum of interval modules
    \begin{align}
        V \cong \! \bigoplus_{I \in \barc (V)} \mathbb F (I),& &W \cong \! \bigoplus_{J \in  \barc (W)} \mathbb F (W) .
    \end{align}
    We can express $ V = V_Y \oplus V_N $, $ W = W_Y \oplus W_N $ denoting
    \begin{align}
        V_Y &\cong \bigoplus_{I \in  \operatorname{coim} \rho} \mathbb F (I),& V_N \cong &\bigoplus_{I \in \barc(V) \backslash \operatorname{coim} \rho} \mathbb F (I),\\
        W_Y &\cong \bigoplus_{J \in  \im \rho} \mathbb F (J),& W_N \cong &\bigoplus_{J \in \barc(J) \backslash \im \rho} \mathbb F (J).
    \end{align}
    The $ V_Y, W_Y $ modules separate the ``yes, matched'' intervals, from the $ V_N, W_N $ ``not matched" intervals. For every interval $ I \in \barc(V_Y) $, $ I $ is $\delta$-matched to an interval $ J \in \barc(W_Y)$ by $\rho(I) = J $. Thus, by Lemma \ref{interval-interleaving-if-matching}, for all pair $ I, J $ of matched intervals, there exist a par of $\delta$-interleaved morphisms
    \begin{equation*}
        \begin{aligned}[c]
        \phi_\delta\colon \mathbb F(I) &\to \mathbb F(J)_\delta\\
        \end{aligned}
        \quad \text{and} \quad
        \begin{aligned}[c]
        \psi_\delta\colon \mathbb F(J) &\to \mathbb F(I)_\delta\\
        \end{aligned}
    \end{equation*}
    which induce the pair of $\delta$-interleaved morphisms
    \begin{equation*}
        \begin{aligned}[c]
        \phi_\delta\colon V_Y &\to {W_Y}_\delta\\
        \end{aligned}
        \quad \text{and} \quad
        \begin{aligned}[c]
        \psi_\delta\colon W_Y &\to {V_Y}_\delta.\\
        \end{aligned}
    \end{equation*}
    Not matched intervals are of length smaller than $ 2 \delta $, therefore both, $V_N$ and $V_Y$ are $\delta$-interleaved with the empty set. We can now construct the $\delta$-interleaving morphism $ \phi\colon V \to W$ such as $\phi \vert_{V_Y} = \phi_Y$ and $\phi \vert_{V_N} = 0$ and, in a similar way, we also construct $ \psi\colon W \to V$.
\end{proof}

To prove the second claim we need several previous lemmas from were we will build a $\delta$-matching from a $\delta$-interleaving morphism. Lets first introduce sue notation.

Let $(V, \pi)$, $(W, \theta)$ be two persistence modules. If $ I = (b, d]$ is an interval with $ d \in \R \cup \{+ \infty\}$, denote $ \barc_{I-}(V) = \left\{ (a, b] \in \barc(V) \colon a \leq b\right\}$. Analogously, we can denote $ \barc_{I+}(V) = \left\{ (b, c] \in \barc(V) \colon c \geq d\right\}$. Let $ \# $ denote the cardinal operator.

\begin{lemma}  \cite[Proposition 3.1.1]{polterovich} \label{lemma:inj-cardinalities}
    Let $ I = (b, d] $ be an interval. It exists an injective morphism $i\colon (V, \pi) \in (W, \theta) $, then $\# \barc_{I-}(V) \leq \# \barc_{I-}(W) $.
\end{lemma}
\begin{proof}
    Let $ E_{I-} = \bigcap_{b < s < d} \im \pi_{s\leq d} \cap \bigcap_{r> d} \ker \pi_{d \leq r} \subseteq V_d $ de the set of elements in $ V_d $ witch come from all $ V_s $ and disappear in all $ V_r $, for $ b < s < d < r $. Thus $ \dim E_{I-}(V) = \# \barc_{I-}(V) $. For every morphism $ p\colon (V, \pi) \to (W, \theta) $ the following diagram conmutes
    $$
    \begin{tikzcd}
        V_s \arrow[r, "\pi_{s \leq r}"] \arrow[d, "p_s"'] & V_r \arrow[d, "p_r"] \\
        W_s \arrow[r, "\theta_{s \leq r}"']               & W_r
    \end{tikzcd}
    $$
    This implies that $ p_r(\im \pi_{s \leq r}) \subseteq \im \theta_{s \leq r} $ and $ p_r(\ker \pi_{s \leq r}) \subseteq \ker \theta_{s \leq r} $. Taking $ r = d $, $ b < s < d $ in the first inclusion, and $ s = d $, $ r > d $ in the second, it happens that $ p_d(E_{I-}(V)) \subseteq E_{I-}(W) $. If we now take $ p = i $, the injective morphism of the hypothesis, we get $ \dim E_{I-}(V) \leq \dim E_{I-}(W)$.
\end{proof}
 
\begin{lemma} \cite[Exercise 3.1.3]{polterovich} \label{lemma:sur-cardinalities}
    Let $ I = (b, d] $ be an interval. It exists a surjective morphism $s\colon (V, \pi) \to (W, \theta) $, then $\# \barc_{I+}(V) \geq \# \barc_{I+}(W) $.
\end{lemma}
\begin{proof} (TO DO)
    
\end{proof}

To build our $\delta$-matching we first define two induced matchings, by an injection and by a surjection respectively. First, suppose that there exists an injection $\iota \colon V \to W$. For every $ c \in \R \cup \{\infty\}$, sort the bars $(a_i, c] \in \barc(V) $, $i \in \{1, \dots, k\} $ by decreasing length order,
$$
    (a_1, c] \supseteq (a_2, c] \supseteq \cdots \supseteq (a_k, c] \text{, with } a_1 \leq a_2 \leq \dots \leq a_k.
$$
Sort in the same manner the bars $(b_j, c] \in \barc(V) $, $j \in \{1, \dots, l\} $,
$$
    (b_1, c] \supseteq (b_2, c] \supseteq \cdots \supseteq (a_k, c] \text{, with } b_1 \leq b_2 \leq \dots \leq b_k.
$$
As there is an injection between $ V $ and $ W$, Lemma \ref{lemma:inj-cardinalities} asures that the amount of bars in $ \barc(V) $ is lower that the amount in $ \barc(V) $, i.e., $ k \leq l $. We define the {\it injective induced matching} $\mu_{inj}\colon \barc(V) \to \barc(W) $ matching, for each $ c \in \R \cup \{\infty\} $, the intervals from both lists by decreasing length.

\begin{lemma} \cite[Proposition 3.1.5]{polterovich} \label{lemma:inj-matching}
    If there exists an injection $\iota\colon (V, \pi) \in (W, \theta) $, then the induced matching $ \mu_{inj}\colon \barc(V) \to \barc (W) $ satisfies:
    \begin{enumerate}
        \item $\operatorname{coim} \mu_{inj} = \barc(V)$, \label{prop:3.1.5. 1}
        \item $\mu_{inj}(a, c] = (b, c], \ \forall b \leq a, \ \forall (a, d] \in \barc(V)$.
    \end{enumerate}
\end{lemma}
\begin{proof}
    Applying Lemma \ref{lemma:inj-cardinalities} with the interval $ (a_k, c] $, we have that for each $ c \in \R \cup \{\infty\} $, $ \# \barc_{(a_k, c]-}(V) \leq \# \barc_{(a_k, c]-}(W) $, having $ k \leq l $ as we note earlier. This means that every bar in $\barc(V)$ is matched to some bar in $\barc(W)$. Hence $\operatorname{coim} \mu_{inj} = \barc(V)$. Moreover, as the matching is carried out in length descending order, for each $ i \in \{1, \dots, k\} $, $\mu_{inj}(a_i, c] = (b_i, c]$, and applying Lemma \ref{lemma:inj-cardinalities}, now with the interval $ (a_i, c] $, and making the same reasoning, $ a_i \leq b_i $.
\end{proof}

Now we suppose that there exists a surjection $ \sigma\colon V \to W $. For every $ a \in \R$, sort the intervals $(a, c_i] \in \barc(V) $, $i \in \{1, \dots, k\} $ by decreasing length order as before,
$$
    (a, c_1] \supseteq (a, c_2] \supseteq \cdots \supseteq (a, c_k] \text{, with } c_1 \geq c_2 \geq \dots \geq a_k,
$$
and again in the same manner, sort the intervals $(a, d_j] \in \barc(V) $, $j \in \{1, \dots, l\} $,
$$
    (a, d_1] \supseteq (a, d_2] \supseteq \cdots \supseteq (a, d_l] \text{, with } d_1 \geq d_2 \geq \dots \geq d_l.
$$
We define the {\it surjective induced matching} $\mu_{sur}\colon \barc(V) \to \barc(W) $ matching, for each $ a \in \R $, the intervals from both lists by decreasing length.

\begin{lemma} \cite[Exercise 3.1.8]{polterovich} \label{lemma:sur-matching}
    If there exists a surjection $s\colon (V, \pi) \to (W, \theta) $, then the induced matching $ \mu_{sur}\colon \barc(V) \to \barc (W) $ satisfies:
    \begin{enumerate}
        \item $\im \mu_{sur} = \barc(W)$,
        \item $\mu_{sur}(a, c] = (a, d], \ \forall a \geq d, \ \forall (a, d] \in \barc(V)$.
    \end{enumerate}
\end{lemma}
\begin{proof}
    Using Lemma \ref{lemma:sur-cardinalities} with the interval $(b, d_k)$ for each $ b \in \R $, we get that, as there exists a surjection between the modules, now $ k \geq l $. Therefore, every bar in $\barc(W)$ is matched to some bar in $\barc(V)$ and $\im \mu_{sur} = \barc(W)$. In an analogue way to the previous lemma, as the intervals in both lists are matched in a decreasing manner, for every $j \in \{1, \dots, l\} $, $\mu_{sur}(a, c_j] = (a, d_j] $, and if we now apply Lemma \ref{lemma:sur-cardinalities}, we get that $ c_j \geq d_j $.
\end{proof}

Hence, with this two matchings at hand, we can define the {induced matching} $ \mu(f)\colon \barc(V) \to \barc(W) $, as the composition $ \mu_{inj} \circ \mu_{sur} $, defined as $ \im \mu_{sur} = \barc(\im f) = \operatorname{coim} \mu_{inj} $.

The following lemma verifies that, in fact, the mapping between persistence modules with its morphisms and barcodes with induced matchings (either the injective or the surjective versions) has functorial properties between the two categories.

\begin{lemma} \cite[Claim 3.1.13]{polterovich} \label{lemma:functoriality}
    Let $ U$, $V$ and $W$ persistence diagrams and $f, g, h$ morphisms between them defined as in the following diagram:
    $$
        \begin{tikzcd}
            U \arrow[r, "f"] \arrow[rr, "h"', bend right] & V \arrow[r, "g"] & W
        \end{tikzcd}.
    $$
    If all $ f, g, h$ are all injections, or all surjections, then the corresponding diagram formed by the barcodes of the modules, and their respective matchings conmutes as well.
    $$
        \begin{tikzcd}
            \barc(U) \arrow[r, "\mu_*(f)"] \arrow[rr, "\mu_*(h)"', bend right] & \barc(V) \arrow[r, "\mu_*(g)"] & \barc(W)
        \end{tikzcd}.
    $$
    Where $ \mu_* $ denotes $ \mu_{inj} $ or $ \mu_{sur} $ accordingly.
\end{lemma}
\begin{proof}
    Let $f, g, h$ injective morphisms, by the definition of the injective induced matching and Lemma \ref{lemma:inj-cardinalities} for any $ d \in \R \cup \{+ \infty\}$, there exist $ k \leq l \leq q $ such that the barcodes of $ U, V, W$ consist on the following bars:
    \begin{align*}
        \barc(U):& (a_1, d] \supset \cdots \supset (a_k, d] \\
        \barc(V):& (b_1, d] \supset \cdots \supset (b_k, d] \supset \cdots \supset (b_l, d] \\
        \barc(V):& (c_1, d] \supset \cdots \supset (c_k, d] \supset \cdots \supset (c_l, d] \supset \cdots \supset (c_q, d]. \\
    \end{align*}
    Therefore, for any $ d $ the diagram conmutes as
    \begin{align*}
        \mu_{inj}(f)(a_i, d] = (b_i, d], \ \mu_{inj}(g)(b_i, d] = (c_i, d], \ \mu_{inj}(h)(a_i, d] = (c_i, d]
    \end{align*}
    for $ 1 \leq i \leq k$. If $f, g, h$ were surjective morphisms, an analogue reasoning using the surjective induced matching definition and Lemma \ref{lemma:sur-cardinalities} completes the proof.
\end{proof}

Finally, we can now claim the two main lemmas from were we will construct our desired $\delta$-matching.

\begin{lemma} \cite[Lemma 3.2.1]{polterovich}
    Let $ (V, \pi), (W, \theta) $ be $\delta$-interleaved persistence modules, with $\delta$-inter\-leaving morphisms $ \phi\colon V \to W_\delta $ and $ \psi\colon W \to V_\delta $. Let $ \phi\colon V \to \im\phi $ be a surjection and $ \mu_{sur}\colon \barc(V) \to \barc (\im \phi)$ the induced matching. Then
    \begin{enumerate}
        \item $\operatorname{coim} \mu_{sur} \supseteq \barc (V)_{\geq 2\delta} $,
        \item $ \im \mu_{sur} = \barc (\im \phi)$ and \label{}
        \item $ \mu_{sur}(b, d] = (b, d'], \ (b, d'] \in \operatorname{coim} \mu_{sur}, \ d' \in [d-2\delta, d]$.
    \end{enumerate}
\end{lemma}
\begin{proof} (INCOMPLETE)
\begin{enumerate}
    \item
    To check the first part, we observe that, in the following diagram, the three morphisms are surjective as $\phi$ and $\pi_{t \leq t+2\delta}$ are defined onto their images, and the diagram conmutes,
    $$
    \begin{tikzcd}
        V \arrow[r, "\phi"] \arrow[rr, "\pi_{t \leq t+2\delta}"', bend right] 
        & \im \phi \arrow[r, "\psi_\delta"] 
        & \im \pi_{t \leq t+2\delta} \\
    \end{tikzcd}.
    $$
    Therefore, because of Lemma \ref{lemma:functoriality} the barcode diagram also conmutes:
    $$
    \begin{tikzcd}
        \barc(V) \arrow[r, "\mu_{sur}(\phi)"] \arrow[rr, "\mu_{sur}(\pi_{t \leq t+2\delta})"', bend right] 
        & \barc(\im \phi) \arrow[r, "\mu_{sur}({\psi_\delta})"] 
        & \barc(\im \pi_{t \leq t+2\delta}) \\
    \end{tikzcd}.
    $$
    By the definition of the surjective induced matching, $ \operatorname{coim} \mu_{sur}(\pi_{t \leq t+2\delta}) = \barc(V)_{\geq 2\delta}$. For each starting point $ a \in \R $, we have that $\barc(\im \pi_{t \leq t+2\delta}) = \{ (a, b-2\delta] \colon (a, b]\in \barc(V), b-a > 2\delta\}$. Sorting all bars of $\barc(V) $ and of $\barc(\im \pi_{t \leq t+2\delta}) $ in length-not-increasing order and matching the bars though the longest-first order, each bar $ (a, b] \in \barc(V) $ is matched with the bar $ (a, b - 2\delta] \in \barc(\im \pi_{t \leq t+2\delta}) $ while $ b -a >2 \delta $. The smaller bars are not matched. Thus, $\operatorname{coim} \mu_{sur}(phi) \supseteq \operatorname{coim} \mu_{sur}(\im \pi_{t \leq t+2\delta}) = \barc (V)_{\geq 2\delta} $

    \item
    The second part is just a reformulation of Lemma \ref{lemma:inj-cardinalities}.

    \item
    The third part
    $$
    \begin{tikzcd}
    \barc(V)_{\geq 2 \delta}
    & \barc(\im \phi)
    & \barc(\im \pi_{t \leq t+2\delta}) \\
    (b, d] \arrow[r, mapsto, "\mu_{sur}(\phi)"] \arrow[u, phantom, "{\rotatebox{90}{$\in$}}"] \arrow[rrd, mapsto, bend right, "\mu_{sur}(\pi_{t \leq t+2\delta})" ]
    & (b, d'] \arrow[r, mapsto, "\mu_{sur}(\psi_\delta)"] \arrow[u, phantom, "{\rotatebox{90}{$\in$}}"]
    & (b, d''] \arrow[u, phantom, "{\rotatebox{90}{$\in$}}"] \\
    &
    & (b, d - 2 \delta] \arrow[u, phantom, "{\rotatebox{90}{$=$}}"]
    \end{tikzcd}
    $$
\end{enumerate}
\end{proof}

\begin{lemma} \cite[Proposition 3.2.2]{polterovich}
    Let $ (V, \pi), (W, \theta) $ be $\delta$-interleaved persistence modules, with $\delta$-inter\-leaving morphisms $ \phi\colon V \to W_\delta $ and $ \psi\colon W \to V_\delta $. Let $ \phi\colon V \to \im\phi $ be a injection and $ \mu_{inj}\colon \barc(\im \phi) \to \barc (W_\delta)$ the induced matching. Then
    \begin{enumerate}
        \item $\operatorname{coim} \mu_{sur} = \barc (\im \phi) $,
        \item $ \im \mu_{inj} \supseteq \barc (W_\delta)_{\geq 2\delta} $ and
        \item $ \mu_{inj}(b, d'] = (b', d'], \ (b, d'] \in \operatorname{coim} \mu_{inj}, \ b' \in [b-2\delta, b]$.
    \end{enumerate}
\end{lemma}
\begin{proof} (INCOMPLETE)
\begin{enumerate}
    \item 
    \item 
    $$
    \begin{tikzcd}
        W \arrow[r, "\psi"] \arrow[rr, "\theta_{t \leq t+2\delta}", bend right] 
        & \im \psi \arrow[r, "\phi_\delta"] 
        & W_{2\delta} \\
    \end{tikzcd}.
    $$

    $$
    \begin{tikzcd}
        \im \theta_{t \leq t+2\delta} \arrow[r, "j"] \arrow[rr, "\theta_{t \leq t+2\delta}"', bend right] 
        & \im \phi_\delta \arrow[r, "i"] 
        & W_{2\delta} \\
    \end{tikzcd}.
    $$

    $$
    \begin{tikzcd}
        \barc(\im \theta_{t \leq t+2\delta}) \arrow[r, "\mu_{inj}(j)"] \arrow[rr, "\mu_{inj}(\theta_{t \leq t+2\delta})"', bend right] 
        & \barc(\im \phi_\delta) \arrow[r, "\mu_{inj}(i)"] 
        & \barc(W_{2\delta}) \\
    \end{tikzcd}.
    $$
    \item 
\end{enumerate}
\end{proof}

\begin{proposition} \cite[Theorem 3.0.2]{polterovich} \label{prop:matching-if-interleaving}
    Given two persistence modules $ V $, $ W $, with a $\delta$-interleaving morphism between them, then there is a $ \delta$-matching between their barcodes.
\end{proposition}
\begin{proof} (INCOMPLETE)
    $$
    \begin{tikzcd}[column sep=large, row sep=large]
        \barc(V)
        &
        & \barc(W[\delta])_{\geq 2 \delta} 
        & \barc(W)_{\geq 2 \delta} \\
        \barc(V)_{\geq 2\delta} \arrow[r, "\mu_{sur}"] \arrow[u, phantom, "{\rotatebox{90}{$\subseteq$}}" description]
        & \barc(\im f) \arrow[r, "\mu_{inj}"]
        & \im \mu_{inj} \arrow[r, "\Psi_\delta"] \arrow[u, phantom, "{\rotatebox{90}{$\supseteq$}}" description]
        & \barc{B}(W) \arrow[u, phantom, "{\rotatebox{90}{$\supseteq$}}" description]\\
        (b, d] \arrow[r, mapsto] \arrow[u, phantom, "{\rotatebox{90}{$\in$}}"]
        & (b, d'] \arrow[r, mapsto] \arrow[u, phantom, "{\rotatebox{90}{$\in$}}"]
        & (b', d'] \arrow[r, mapsto] \arrow[u, phantom, "{\rotatebox{90}{$\in$}}"] 
        & (b' + \delta, d' + \delta] \arrow[u, phantom, "{\rotatebox{90}{$\in$}}"]
    \end{tikzcd}
    $$
\end{proof}

\begin{theorem}[Stability] \cite[Theorem 2.2.8]{polterovich}
    There is an isometry between a persistence module with the interleaving distance and its barcode with the bottleneck distance. This means that, given two persistence modules $ V, \ W $, 
    $$ 
        d_{int} (V, W) = d_{bot} (\barc(V), \barc(W)).
    $$
\end{theorem}
\begin{proof}
    Suppose $ d_{int}(V, W) = \delta $. Proposition \ref{prop:matching-if-interleaving} asures there exist a $\delta$-matching between $ \barc(V) $ and $ \barc (W) $. As $ d_{bot}(V, W) $ is the infimum $\delta$ for witch exists a $\delta$-matching, $ d_{bot}(V, W) \leq d_{int}(V, W)$. On the other hand, Proposition \ref{prop:interleaving-if-matching} leads, with the same reasoning, to $ d_{int}(V, W) \leq d_{bot}(V, W)$. Thus, it has to be $ d_{int} (V, W) = d_{bot} (\barc(V), \barc(W)) $.
\end{proof}
